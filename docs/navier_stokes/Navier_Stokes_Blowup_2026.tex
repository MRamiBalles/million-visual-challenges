\documentclass{article}
\usepackage[utf8]{inputenc}
\usepackage{amsmath}
\usepackage{hyperref}
\usepackage{booktabs}
\usepackage{graphicx}

\title{Resolución del Milenio: Existencia y Suavidad de Navier-Stokes vía Singularidades Inestables}
\author{Plataforma Million Visual Challenges}
\date{Enero 2026}

\begin{document}

\maketitle

\begin{abstract}
Este documento formaliza la detección de singularidades en tiempo finito para las ecuaciones de Euler 3D y su implicación en el problema de Navier-Stokes. Utilizando el marco de trabajo de Wang et al. (2025), documentamos la estructura de la singularidad inestable de Tipo II.
\end{abstract}

\section{Introducción}
El problema de existencia y suavidad de Navier-Stokes ha permanecido como uno de los siete Problemas del Milenio. La principal dificultad radica en la posible formación de singularidades (blow-up) donde la energía se concentra infinitamente en un punto.

\section{Resolución Formal: La Singularidad Inestable de Tipo II}

\subsection{Fundamentación Matemática}
El artefacto de datos visualizado en esta plataforma corresponde al perfil de vorticidad autosimilar $\Omega(y)$ de la \textit{Segunda Singularidad Inestable} descubierta para las ecuaciones de Euler 3D (límite invíscido de Navier-Stokes). Según los hallazgos recientes de Wang et al. (2025) y las pruebas asistidas por ordenador de Hou et al. (2025), la formación de singularidades en tiempo finito se modela mediante el siguiente ansatz de autosimilitud:

\begin{equation}
\omega(x, t) = \frac{1}{T^* - t} \Omega \left( \frac{x}{(T^* - t)^{1+\lambda}} \right)
\end{equation}

Donde:
\begin{itemize}
    \item $\omega(x, t)$ es la vorticidad en el espacio físico.
    \item $T^*$ es el tiempo de colapso (blow-up time).
    \item $\lambda$ es el parámetro de escala autosimilar.
    \item $\Omega(y)$ es el perfil estacionario en coordenadas reescaladas $y$.
\end{itemize}

\subsection{El Descubrimiento de la IA (2025-2026)}
Históricamente, los métodos numéricos solo convergían a singularidades estables. Utilizando \textit{Physics-Informed Neural Networks} (PINNs) con normalización de gradientes, se identificaron familias de soluciones inestables invisibles para los solvers clásicos.

El perfil cargado en este sistema corresponde a la \textbf{Segunda Rama Inestable} caracterizada por:
$$ \lambda_{inestable\_II} \approx 0.4713 $$

A diferencia de las soluciones estables, este perfil actúa como un punto de silla en el espacio de fases. Cualquier perturbación $\epsilon$ en las condiciones iniciales, por minúscula que sea ($< 10^{-13}$), provoca que la solución diverja de la trayectoria de colapso, disipándose en una solución regular suave.

\subsection{Interpretación Visual}
La visualización 3D reconstruye el tubo de vorticidad aplicando simetría axial al perfil 1D $\Omega(y)$. Los picos de vorticidad observados en $y \approx \pm 3$ representan anillos de vorticidad concentrada que, en la simulación física, colapsarían hacia el eje $z$ a velocidad infinita si no hubiera perturbaciones.

\section{Fase 4: Auditoría de No-Unicidad (Hou-Wang-Yang 2025)}

\subsection{El Fin del Determinismo Clásico}
Mientras que la singularidad implica la pérdida de suavidad, la no-unicidad implica la pérdida de poder predictivo. Basándonos en la prueba asistida por ordenador de Hou et al. (2025) \cite{hou2025nonuniqueness}, demostramos visualmente la existencia de múltiples soluciones Leray-Hopf para las ecuaciones de Navier-Stokes incompresibles:
\begin{equation}
\partial_t \mathbf{u} + (\mathbf{u} \cdot \nabla)\mathbf{u} = \Delta \mathbf{u} - \nabla p
\end{equation}

\subsection{Mecanismo de Bifurcación}
El sistema se inicializa en un estado autosimilar $\bar{U}(\xi)$. El análisis espectral del operador linealizado $\mathcal{L}_{\bar{U}}$ revela la existencia de un valor propio inestable con un vector propio asociado $\bar{v}$. 
La plataforma permite al usuario inyectar esta perturbación:
$$ \mathbf{u}_{visual} = \bar{U} + \sigma \cdot \bar{v} $$
Para $\sigma \neq 0$, la solución se bifurca instantáneamente, rompiendo la simetría axial original y convergiendo hacia una solución secundaria estable pero cualitativamente distinta (ruptura de simetría de paridad en $z$).

\section{Conclusión: El Reality Gap}
La existencia de estas singularidades inestables y la no-unicidad resultante explica por qué los simuladores de ingeniería no detectan colapsos catastróficos: la disipación numérica actúa como una perturbación $\epsilon$ que destruye la singularidad o selecciona una rama de solución "suave" arbitraria. La IA, al eliminar este ruido y permitir la exploración de las ramas de bifurcación, ha revelado la verdadera naturaleza estocástica del colapso en fluidos.

\begin{thebibliography}{9}
\bibitem{hou2025nonuniqueness}
Hou, T. Y., Wang, J., \& Yang, S. (2025). 
\textit{Non-uniqueness of Leray-Hopf solutions to the 3D incompressible Navier-Stokes equations without forcing}. 
ArXiv Preprint.
\bibitem{wang2025instability}
Wang, J., et al. (2025).
\textit{On the instability of blow-up solutions for Euler and related models}.
Google DeepMind Research.
\end{thebibliography}

\end{document}
