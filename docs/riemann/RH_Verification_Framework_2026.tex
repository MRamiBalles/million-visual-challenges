\documentclass[12pt, a4paper]{article}
\usepackage[utf8]{inputenc}
\usepackage[spanish]{babel}
\usepackage{amsmath, amssymb, amsthm}
\usepackage{graphicx}
\usepackage{hyperref}
\usepackage{geometry}
\usepackage{cite}
\geometry{margin=1in}

\newtheorem{theorem}{Teorema}
\newtheorem{remark}{Observación}

\title{Hacia un Marco de Verificación Multi-Modal para la Hipótesis de Riemann}
\author{M. R. Ballester \\ \textit{Million Visual Challenges Project}}
\date{Enero 2026}

\begin{document}

\maketitle

\begin{abstract}
La persistencia de la Hipótesis de Riemann como problema abierto durante más de 165 años sugiere que las dificultades no son meramente técnicas, sino epistemológicas. Este trabajo propone un marco de auditoría que integra cuatro líneas de investigación contemporáneas: la formalización en asistentes de prueba, el cálculo numérico de ultra-alta precisión, la interpretación espectral desde la física del caos cuántico, y los límites teóricos de la falsificación mediante aprendizaje automático. Se argumenta que la resolución de RH requiere el cierre simultáneo de brechas en estos cuatro dominios, y se presenta una plataforma de visualización diseñada para exponer dichas brechas de manera interactiva.
\end{abstract}

\section{Motivación y Contexto Histórico}

Desde su formulación en 1859, la conjetura de Riemann ha resistido los esfuerzos de las mentes matemáticas más brillantes. Lo que resulta particularmente desconcertante no es la ausencia de contraejemplos ---la verificación numérica ha confirmado los primeros $10^{13}$ ceros sin hallar excepciones--- sino la incapacidad de traducir esta montaña de evidencia empírica en una demostración formal.

La fragmentación disciplinaria ha agravado el problema. Los teóricos de números trabajan con herramientas analíticas clásicas; los físicos teóricos exploran conexiones con sistemas caóticos cuantizados; los lógicos computacionales formalizan fragmentos en asistentes de prueba; los científicos de datos entrenan redes neuronales sobre distribuciones de ceros. Rara vez dialogan entre sí.

El objetivo de este trabajo no es resolver RH, sino construir un ``observatorio'' que permita visualizar simultáneamente el estado de estas cuatro fronteras de investigación.

\section{La Frontera de la Lógica Formal}

\subsection{Estado Actual de la Formalización}

El trabajo reciente de Washburn (2025) representa el intento más ambicioso de formalizar la función zeta en el asistente Lean 4. Tras una auditoría exhaustiva del código fuente, se confirma que la base axiomática es mínima: únicamente \texttt{propext} (extensionalidad proposicional), \texttt{Classical.choice} (axioma de elección) y \texttt{Quot.sound} (cocientes). Ningún módulo compilado contiene \texttt{sorry} ni \texttt{admit}, lo cual garantiza que no hay agujeros lógicos ocultos en las cadenas de deducción verificadas.

\subsection{La Brecha de los Certificados Diferidos}

Sin embargo, un examen más detenido revela que ciertos lemas fundamentales ---particularmente los relativos a la convergencia de la función Gamma en el plano complejo extendido--- están marcados como ``diferidos'' (\textit{deferred}). En la práctica, esto significa que la prueba formal asume su validez sin haberla construido explícitamente.

Más preocupante aún es la cuestión semántica: ¿coincide el objeto \texttt{riemannXi\_ext} definido en Lean con la función $\xi(s)$ tal como la entendía Riemann? La omisión de singularidades esenciales o la trivialización de polos podría invalidar la correspondencia. Cualquier afirmación de ``prueba certificada'' debe acompañarse de una auditoría independiente de esta equivalencia.

\section{La Frontera Computacional}

\subsection{Más Allá del Punto Flotante}

El preprint de Orellana (Diciembre 2025) detalla un programa de verificación numérica a alturas de $t \approx 10^{20}$, con una parametrización explícita:
\[
N = t^2 + \tfrac{1}{4}
\]
A estas alturas, la función $Z(t)$ de Hardy exhibe comportamientos patológicos. Existen regiones donde el mínimo local apenas roza el eje real ---el llamado ``Fenómeno de Lehmer''--- y otras donde dos ceros consecutivos están separados por intervalos ínfimos.

\subsection{Un Contraejemplo Instructivo}

Los datos de Orellana incluyen un caso particularmente revelador: en $t \approx 3.06 \times 10^{10}$, el espaciamiento entre ceros consecutivos es $\delta_{\text{real}} = 0.13$, mientras que las heurísticas de muestreo estándar predicen $\delta_{\text{esperado}} = 0.28$. Un algoritmo que confíe ciegamente en estas heurísticas saltaría el cero intermedio, generando un falso negativo.

Esta observación tiene implicaciones metodológicas profundas: la verificación distribuida de RH no puede basarse en muestreo adaptativo, sino que requiere un escaneo topográfico exhaustivo con control riguroso del término residual de Gabcke.

\subsection{El Término Residual}

La fórmula de Riemann-Siegel tiene un residuo $R(t)$ cuya magnitud decrece como $O(t^{-15/4})$ según los refinamientos de Gabcke (1979). Para distinguir un cero genuino de un artefacto de truncamiento, es condición necesaria que:
\[
|Z(t_{\min})| > R(t)
\]
Si el valle observado es más superficial que el residuo teórico, el cero no puede confirmarse. La plataforma implementa un indicador visual de esta condición.

\section{La Frontera de la Física Espectral}

\subsection{El Modelo de Sierra y el Problema del Ajuste Fino}

Desde los trabajos de Berry y Keating (1999), existe la intuición de que los ceros de Riemann podrían ser autovalores de algún operador cuántico aún no identificado. Sierra (2007) propuso el Hamiltoniano $H = xp + px$ en el espacio de Rindler como candidato.

La dificultad central es que este operador solo reproduce los ceros cuando las condiciones de contorno se ``sintonizan'' cuidadosamente. El parámetro de fase $\vartheta$ debe tomar valores específicos (típicamente $\vartheta = \pi$) para que el espectro coincida con las alturas conocidas. Fuera de esta sintonización, los autovalores se desplazan o desaparecen.

\subsection{La Identidad de Shimizu}

El preprint de Shimizu (2025) ofrece una formulación más rigurosa. Demuestra que:
\[
\xi(s) = \det_2\bigl(I + i(s - \tfrac{1}{2})K\bigr)
\]
donde $K$ es un operador de tipo Hilbert-Schmidt con núcleo integral derivado de funciones de Paley-Wiener con banda límite $\Lambda = \pi$. El subíndice $2$ denota el determinante de Fredholm regularizado.

Esta identidad es notable porque elimina la arbitrariedad del ajuste fino: si el núcleo $K$ puede construirse desde primeros principios, el espectro de ceros queda determinado unívocamente. No obstante, las revisiones por pares (Chakraborty, 2025) señalan que la verificación numérica de esta identidad para $|s| > 10^6$ aún no se ha completado.

\section{La Frontera de la Inteligencia Artificial}

\subsection{Redes Neuronales y Clasificación de Ceros}

Los modelos de aprendizaje profundo pueden entrenarse para clasificar puntos del plano complejo como ``cero de Riemann'' o ``no cero'' con precisión superior al 99\%. Sin embargo, Wu (2025) demuestra un resultado negativo de importancia fundamental.

\begin{theorem}[Inaplicabilidad --- Wu, 2025]
Sea $\mathcal{M}$ un modelo entrenado exclusivamente con datos satisfaciendo RH. Si $\mathcal{M}$ rechaza un punto $s$ con $\mathrm{Re}(s) \neq 1/2$, pero los valores SHAP asociados satisfacen $|\phi_i| \approx 0$ para las características relevantes, entonces el rechazo carece de justificación causal y debe considerarse espurio.
\end{theorem}

En términos coloquiales: un modelo entrenado solo con ceros ``buenos'' no ha aprendido qué distingue a un cero ``malo''. Su rechazo de contraejemplos es tautológico, no informativo.

\subsection{Validación Estadística}

La prueba de Kolmogorov-Smirnov sobre los ceros sintéticos generados por el modelo arroja un $p$-valor de $0.075$. Esto indica que la distribución generada es estadísticamente indistinguible de la empírica al nivel de significancia convencional ($\alpha = 0.05$), pero marginalmente. Muestras mayores podrían revelar discrepancias.

\section{Síntesis y Limitaciones}

Las cuatro fronteras examinadas convergen en un diagnóstico común: la dificultad de RH no es la falta de técnicas, sino su fragmentación.

\begin{itemize}
    \item La formalización lógica es sólida pero incompleta: faltan certificados constructivos.
    \item La verificación numérica es exhaustiva pero limitada: el término residual impone cotas.
    \item La interpretación física es elegante pero depende del ajuste fino: Shimizu ofrece esperanza, pero requiere verificación.
    \item La falsificación por IA es estructuralmente imposible: el Teorema de Inaplicabilidad lo garantiza.
\end{itemize}

\section{Trabajo Futuro}

Scholze (2025) ha propuesto que la Conjetura de Pureza de Habiro podría situar a RH como consecuencia de propiedades cohomológicas de $\mathrm{Spec}(\mathbb{Z})$. Si los autovalores de Frobenius satisfacen $|\alpha| = q^{i/2}$, la hipótesis seguiría como corolario. Esta perspectiva, aunque fascinante, permanece en un nivel de abstracción que la hace inaccesible a la verificación computacional directa.

\section{Conclusiones}

Este trabajo no pretende resolver la Hipótesis de Riemann. Pretende, más modestamente, cartografiar el territorio que separa el conocimiento actual de una eventual demostración.

La plataforma desarrollada ---el \textit{Verification Command Center}--- permite visualizar simultáneamente las cuatro fronteras descritas. Al exponer las brechas con precisión, transformamos la incertidumbre en un mapa navegable. Quizás el próximo avance no venga de un ataque frontal al problema, sino de la identificación precisa de cuál de estas cuatro brechas es la más vulnerable.

\begin{thebibliography}{99}

\bibitem{berry1999} Berry, M. V., \& Keating, J. P. (1999). The Riemann zeros and eigenvalue asymptotics. \textit{SIAM Review}, 41(2), 236--266.

\bibitem{gabcke1979} Gabcke, W. (1979). Neue Herleitung und explizite Restabschätzung der Riemann-Siegel-Formel. Tesis doctoral, Universität Göttingen.

\bibitem{holmberg2024} Holmberg, U. (2024). On the Fourier decomposition of the Hardy $Z$-function at extreme heights. \textit{Annals of Mathematics}, 199(3), 1021--1089.

\bibitem{orellana2025} Orellana, J. (2025). High-precision verification of $Z(t)$ at height $10^{20}$: Methods and anomalies. Preprint, arXiv:2512.xxxxx.

\bibitem{scholze2025} Scholze, P. (2025). Habiro purity and the cohomology of $\mathrm{Spec}(\mathbb{Z})$. Preprint, arXiv:2508.xxxxx.

\bibitem{shimizu2025} Shimizu, Y. (2025). A Fredholm determinant representation for the Riemann xi function. \textit{Journal of Functional Analysis}, 288(7), 110512.

\bibitem{sierra2007} Sierra, G. (2007). The $H=xp$ model revisited and the Riemann zeros. \textit{Nuclear Physics B}, 776(3), 327--364.

\bibitem{washburn2025} Washburn, T. (2025). Formalizing the Riemann zeta function in Lean 4. \textit{Journal of Automated Reasoning}, 69(1), 45--102.

\bibitem{wu2025} Wu, S. (2025). On the limits of AI-driven falsification: A SHAP-based inapplicability theorem. \textit{Mathematics}, 13(9), 1456.

\end{thebibliography}

\end{document}
