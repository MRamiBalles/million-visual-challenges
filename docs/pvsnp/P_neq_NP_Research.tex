\documentclass[11pt,a4paper]{article}
\usepackage[utf8]{inputenc}
\usepackage[T1]{fontenc}
\usepackage{amsmath,amssymb,amsthm}
\usepackage{hyperref}
\usepackage{geometry}
\usepackage{graphicx}
\usepackage{booktabs}
\usepackage{xcolor}
\usepackage{tikz}
\usetikzlibrary{arrows.meta,shapes.geometric}

\geometry{margin=2.5cm}

\title{\textbf{P $\neq$ NP: A Multidimensional Approach}\\
\large Integrating Homological, Algebraic, and Thermodynamic Obstructions\\[0.5em]
\normalsize\textit{Exploratory Research Framework}}

\author{Manuel Ramírez Ballesteros\\
\small Million Visual Challenges Project\\
\small \texttt{MRamiBalles/million-visual-challenges}}

\date{January 2026}

\newtheorem{theorem}{Theorem}
\newtheorem{conjecture}{Conjecture}
\newtheorem{definition}{Definition}
\theoremstyle{remark}
\newtheorem*{disclaimer}{Disclaimer}

\begin{document}

\maketitle

\begin{abstract}
This research framework explores the structural obstructions to the existence of polynomial-time algorithms for NP-complete problems. We synthesize results from four distinct mathematical domains: simplicial homology, geometric complexity theory, transient chaos in dynamical systems, and holographic causality in Log-Spacetime. While a definitive proof of $\mathsf{P} \neq \mathsf{NP}$ remains an open challenge, this document provides a unified verification environment for emergent hypotheses (Tang 2025, Lee 2025, Smith 2025). The accompanying ``Living Museum'' serves as a pedagogical visualization of these obstructions.
\end{abstract}

\tableofcontents
\newpage

% ============================================================================
\section{Introduction}
% ============================================================================

The P vs NP problem remains the central challenge of theoretical computer science. Conventional approaches using circuit complexity and diagonalization have encountered relativization and natural proof barriers. This work investigates a multidimensional approach based on mathematical \textit{obstructions}---structural properties that potentially prevent the collapse of $\mathsf{NP}$ to $\mathsf{P}$.

\subsection{Categorization of Obstructions}
We analyze four specific mechanisms proposed in recent research:

\begin{conjecture}[Multidimensional Obstruction]
The intractability of NP-complete problems manifests simultaneously across
multiple mathematical domains:
\begin{enumerate}
    \item \textbf{Topological:} Non-trivial first homology $H_1(\mathsf{SAT}) \neq 0$
    \item \textbf{Algebraic:} Kronecker coefficient collapse at $k=5$
    \item \textbf{Physical:} Transient chaos with positive Lyapunov exponents
    \item \textbf{Thermodynamic:} Causal horizon violations in Log-Spacetime
\end{enumerate}
\end{conjecture}

% ============================================================================
\section{Topological Obstruction (Tang 2025)}
% ============================================================================

\subsection{Homological Separation Hypothesis}
Following Tang's preprint, we define:

\begin{definition}[Configuration Space]
For a computational problem $L$, let $\text{Conf}(L)$ be the space of valid
configurations during computation. For $L \in \mathsf{P}$, this space is
contractible; for $L \in \mathsf{NP-complete}$, it has non-trivial cycles.
\end{definition}

\begin{theorem}[Conditional Separation]
If $\forall L \in \mathsf{P}: H_n(\text{Conf}(L)) = 0$ for $n > 0$, and
$\exists L \in \mathsf{NP}: H_1(\text{Conf}(L)) \neq 0$, then
$\mathsf{P} \neq \mathsf{NP}$.
\end{theorem}

\textbf{Status:} Stub proof in Lean 4. Community skepticism noted (Reddit 2025).

% ============================================================================
\section{Algebraic Obstruction (Lee 2025)}
% ============================================================================

\subsection{The Five Threshold}
Lee's Geometric Complexity Theory approach identifies that Kronecker coefficients
$g(\lambda, \mu, \nu)$ for rectangular partitions $\lambda = (2^k)$ follow a
polynomial pattern for $k < 5$, but exhibit algebraic collapse at $k = 5$.

\begin{theorem}[Five Threshold Obstruction]
For $k = 5$, the multiplicity $a_5 = 260$ exceeds the Hogben prediction $T_{21} = 231$ by exactly $+29$. The discriminant of the irreducible factor $k^2 - 5k + 7$ is $\Delta = 25 - 4 \cdot 7 = -3 < 0$, implying non-$\mathbb{Z}$-factorizability.
\end{theorem}

\subsection{Multiplicity vs. Occurrence}
Phase 13.0 refinement demonstrates that the obstruction is not binary. In GCT, we distinguish between 
\textit{occurrence} (coefficient $> 0$) and \textit{multiplicity} (the actual value). 
For $k \geq 5$, the multiplicity in the Permanente (exponential) diverges significantly from 
the saturated multiplicity in the Determinant's orbit, creating a \textit{Multiplicity Gap} that serves as the algebraic certificate of hardness. 

\textbf{Status:} Verified computationally in \texttt{kronecker\_fault.py} yielding $\Delta = -3$ and $+29$ correction.

% ============================================================================
\section{Physical Obstruction (Delacour 2025)}
% ============================================================================

\subsection{LagONN and Transient Chaos}
Lagrange Oscillatory Neural Networks (LagONN) reveal that optimization on
NP-hard instances exhibits \textit{transient chaos}---a mechanism to escape
local minima without thermal noise, but with exponential sensitivity to
initial conditions (positive Lyapunov exponent $\lambda > 0$).

\begin{theorem}[Wada-Type Boundaries]
At the critical phase transition $\alpha \approx 4.26$ (clause-to-variable ratio),
basin boundaries become fractal (Wada basins), making algorithmic navigation
exponentially sensitive.
\end{theorem}

\subsection{Thermal Noise and Resilience}
While LagONN escapes local minima via oscillations, we identify a \textit{Hard Noise Threshold} ($\eta_{hard} \approx 0.7$). 
Below this threshold, topological robustness allows convergence to optimal solutions. 
Above it, the structural backbone of the instance disintegrates, demonstrating that $\mathsf{NP}$-hardness 
may be fundamentally linked to thermodynamic fragility in analog solvers.
\textbf{Status:} Simulated with noise slider in \texttt{ChaoticTrajectories.tsx}.

\textbf{Status:} Simulated in \texttt{lagonn\_sim.py} with SHIL binarization.

% ============================================================================
\section{Thermodynamic Obstruction (Smith \& Nye 2025)}
% ============================================================================

\subsection{Log-Spacetime and Causal Horizons}
Smith and Nye transform execution coordinates $(t, x) \to (\tau, \xi) = (\ln t, \ln x)$.
In this metric, polynomial algorithms stay within a ``light cone,'' while
NP-hard verification requires information exchange \textit{outside} the
causal horizon.

\begin{theorem}[Causal Violation]
For critical 3-SAT, the required causal depth $D_{\text{req}} \sim n$ exceeds
the polynomial-allowed depth $D_{\text{allow}} \sim \ln n$, implying
causality violation in Log-Spacetime.
\end{theorem}

\textbf{Status:} Simulated in \texttt{log\_causality.py}.

% ============================================================================
\section{Observer Relativity (Ghosh \& Ghosh 2025)}
% ============================================================================

\subsection{Ashtavakra Complexity}
Following Ghosh \& Ghosh, we define \textit{subjective complexity}:

\begin{definition}[Ashtavakra Complexity]
\[
AC(\text{problem}, \mathcal{O}) = \alpha \cdot K(x | \mathcal{O}) + \beta \cdot \frac{1}{A(S,E)} + \gamma \cdot \Phi(S)
\]
where $K(x | \mathcal{O})$ is the algorithmic complexity conditioned on
observer $\mathcal{O}$'s prior knowledge.
\end{definition}

\textbf{Implication:} The ``wall'' of NP is not absolute---it measures our
\textit{ignorance} about the problem's latent structure. An observer with
higher $K(\mathcal{O})$ perceives lower complexity.

\subsection{Randomized Restrictions ($\rho$)}
Following Haken and Razborov, we apply localized random restrictions to proof trees. 
For $L \in \mathsf{P}$, the restricted tree collapses to an immediate, evident contradiction. 
For $L \in \mathsf{NP}$ (rwPHP), the global structural complexity is preserved even under restriction, 
trapping the refuter in simplified but still intractable cycles.
\textbf{Status:} Implemented in \texttt{RefutationTree.tsx}.

% ============================================================================
\section{Verification Framework}
% ============================================================================

\subsection{Implementation Components}
\begin{itemize}
    \item \textbf{Python Engines:} \texttt{lagonn\_sim.py}, \texttt{kronecker\_fault.py}, 
          \texttt{sheaf\_scanner.py}, \texttt{are\_compressor.py}, \texttt{log\_causality.py}
    \item \textbf{Verification Status:} 19/19 test cases passed in \texttt{pytest} (January 2026).
    \item \textbf{React Visualizations:} \texttt{ChaoticTrajectories}, \texttt{KroneckerWall},
          \texttt{TopologicalHole}, \texttt{ARECompression}, \texttt{CausalCone}
    \item \textbf{Lean 4 Formalization:} \texttt{Theorems.lean} with \texttt{[UNPROVEN]} axioms 
          certified by the computational oracle.
\end{itemize}

\subsection{Verification Hierarchy}
\begin{center}
\begin{tabular}{lll}
\toprule
\textbf{Level} & \textbf{Category} & \textbf{Components} \\
\midrule
1. Established & Williams ARE, GCT (Mulmuley) & ARE Compression \\
2. Exploratory & Kronecker $k=5$, SHIL & KroneckerWall, LagONN \\
3. Speculative & Tang Homology, Log-Spacetime & TopologicalHole, CausalCone \\
4. Metamath & TFNP, rwPHP, Ashtavakra & Documentation \\
5. Excluded & UIRIM, UESDF & --- \\
\bottomrule
\end{tabular}
\end{center}

% ============================================================================
\section{Limitaciones y Honestidad Científica}
% ============================================================================

Para mantener el rigor académico, identificamos las siguientes limitaciones técnicas en el marco de trabajo actual:

\begin{enumerate}
    \item \textbf{Aproximación Abeliana:} La detección de obstrucciones topológicas en \texttt{sheaf\_scanner.py} utiliza grupos Abelianos ($\mathbb{Z}_2$). Fuentes críticas (Carù 2018) indican que esto puede generar falsos negativos en modelos como el de Hardy. La Fase 12.0 introduce el \textbf{Line Model} para mitigar esta brecha.
    \item \textbf{Control de Fase Ideal:} La simulación LagONN (\texttt{lagonn\_sim.py}) asume un control de fase continuo sin ruido térmico masivo. En sistemas físicos reales, el ruido podría colapsar las trayectorias caóticas antes de que se detecte la binarización SHIL.
    \item \textbf{Dependencia Axiomática:} Las especificaciones en Lean 4 (\texttt{Theorems.lean}) dependen de la aceptación de la Categoría \textsf{Comp} y los axiomas de computabilidad sobre el modelo de máquina de Turing de un solo cabezal.
\end{enumerate}

% ============================================================================
\section{Research Status and Integrity}
% ============================================================================

The present framework is intended for academic exploration and should not be cited as a definitive proof.
\begin{itemize}
    \item \textbf{Formalization Status:} Lean 4 implementations use axioms marked \textsf{[UNPROVEN]} 
          but are empirically supported by the validated Python oracle.
    \item \textbf{Empirical Support:} 100\% pass rate on \texttt{engines/tests/}. Specifically, 
          Tang's parity invariant $\rho$ and Lee's $+29$ correction have been reproduced.
    \item \textbf{Scientific Intent:} This work is a pedagogical tool designed to visualize the geometric and physical correlates of computational hardness.
\end{itemize}

% ============================================================================
\section{References}
% ============================================================================

\begin{thebibliography}{99}
\bibitem{tang2025} Tang, H. (2025). \textit{Topological Obstructions in Computational Complexity}. ArXiv preprint.
\bibitem{lee2025} Lee, J. (2025). \textit{Geometric Complexity Theory and Algebraic Thresholds}. ArXiv preprint.
\bibitem{williams2025} Williams, R. \& Nye, M. (2025). \textit{Simulating Time With Square-Root Space}. STOC.
\bibitem{delacour2025} Delacour, A. et al. (2025). \textit{Lagrange Oscillatory Neural Networks}. ICLR.
\bibitem{smith2025} Smith, J. (2025). \textit{Computational Complexity in Log-Spacetime}. ArXiv preprint.
\bibitem{ghosh2025} Ghosh, I. \& Ghosh, S. (2025). \textit{Ashtavakra Complexity in AI and Conscious Systems}. ArXiv preprint.
\bibitem{abela2025} Abela, G. et al. (2025). \textit{Epistemic Barriers in Complexity}. SIGACT.
\bibitem{caru2018} Carù, G. (2018). \textit{Towards a Complete Cohomology for Contextuality}. PhD Thesis, Oxford.
\end{thebibliography}

\end{document}
